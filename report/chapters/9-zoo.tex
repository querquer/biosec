\section{Doddingtons Zoo}


Doddingtons Zoo
geht es darum dass Sprecher ein unterschiedliches Verhalten bezüglich den Erfolg ihrer Authentifizierung aufzeigen. \cite{zoo}
Dadurch lassen sich Sprecher in vier Kategorien unterscheiden die jeweils von einem Tier repr"asentiert werden. \\

Beschreibung dieser Kategorieren:
\begin{itemize}
	\item[\textbullet]  \textbf{Sheeps:} außerordentlich \textit{leicht} von dem System erkannt, die Mehrheit der Sprecher gehört dieser Kategorie an
	\item[\textbullet]  \textbf{Goats:} außerordentlich \textit{schwer} von dem System erkannt
	\item[\textbullet]  \textbf{Lambs:} außerordentlich \textit{verwundbar} gegenüber Nachahmung
	\item[\textbullet]  \textbf{Wolves:} außerordentlich \textit{erfolgreich} bei der Nachahmung anderen Sprecher
\end{itemize}
\cite{dittmann}\\

Nun sollte eine Projektion dieser Kategorien auf die Ergebnisse unserer Klassifizierung vollzogen werden. Als Entscheidungsgrundlage dienten nun nicht nur die erfolgreich klassifizierten Samples sondern auch die Verteilung der falsch klassifizierten Samples. Um die Projektion umzusetzen haben wir folgendes Schema erarbeitet und auf unsere Datenbasis angewandt.\\

verwendetes Schema zur Kategorisierung der Sprecher:
\begin{itemize}
	\item[\textbullet]  \textbf{Sheeps:} viele richtig klassifizierte Samples
	\item[\textbullet]  \textbf{Goats:} wenig richtig klassifizierte Samples
	\item[\textbullet]  \textbf{Lambs:} viele Sample von anderen Sprechern wurden diesem Sprecher zugeordnet
	\item[\textbullet]  \textbf{Wolves:} viele Samples bei wenigen anderen Sprechern zugeordnet
\end{itemize}

Dieser Ansatz wurde auf das Ergebnis der Klassifikation mit dem Klassifikator IBK auf das \textit{mixed test set} angewandt. Das Ergebnis dieser Projektion ist zu sehen in Tabelle:\ref{table:resultsZoo}.


\begin{table}[h]
\tbl{Ergebnisse der Kategorisierung nach Doddingtons Zoo}{%
\begin{tabular}{ | l | l | l | l | l |}
    \hline
    Animal & female & male & mixed & Anteil \\ \hline 
    Sheep 	& 32 	& 43	&	75	&	90.36\%	\\ 
    Goat	& 2		& 2		& 4 	&	4.82\% \\ 
    Lamb	& 1  	& 1		& 1		&	2.41\% \\ 
    Wolf	& 0  	& 2 	& 2		&	2.41\% \\ \hline
\end{tabular}}
\label{table:resultsZoo}
\end{table}


Das Ergebnis der kategorisierung zeigt dass der großteil unsere Sprecher \textbf{Sheeps} sind.
Wie zu erwarten wurden unsere falsch authentifizierten Sprecher zu der Kategorie der \textbf{Goat} zugeordnet.




\section{Doddingtons Zoo}


Doddingtons Zoo
geht es darum dass sprecher ein unterschiedliches verhalten bezüglich den erfolg ihrer authentifikation aufzeigen. \cite{zoo}
dabei lassen sich sprecher in vier kategoriern unterscheiden die jeweils von einem tier representiert werden. \\

Beschreibung dieser Kategorieren:
\begin{itemize}
	\item[\textbullet]  \textbf{Sheeps:} außerordentlich \textit{leicht} von dem System erkannt, die Mehrheit der Sprecher gehört dieser Kategorie an
	\item[\textbullet]  \textbf{Goats:} außerordentlich \textit{schwer} von dem System erkannt
	\item[\textbullet]  \textbf{Lambs:} außerordentlich \textit{verwundbar} gegenüber Nachahmung
	\item[\textbullet]  \textbf{Wolves:} außerordentlich \textit{erfolgreich} bei der Nachahmung anderen Sprecher
\end{itemize}
\cite{dittmann}


nun war es die aufgabe war nun eine projektion dieser kategorien auf unsere klassifizierungs ergebnisse. als entscheidungsgrundlage waren nun nicht nur  die erfolgreich klassifizierten samples intressant sondern auch die verteilung der samples die nicht dem richigen sprecher zugeordnet wurden. Um die projektion umzusetzen haben wir folgendes schema erstellt und angewandt auf unsere Datenbasis.\\

verwendetes Schema zur Kategorisierung der Sprecher:
\begin{itemize}
	\item[\textbullet]  \textbf{Sheeps:} viele richtig klassifizierte Samples
	\item[\textbullet]  \textbf{Goats:} wenig richtig klassifizierte Samples
	\item[\textbullet]  \textbf{Lambs:} viele Sample von anderen Sprechern wurden diesem Sprecher zugeordnet
	\item[\textbullet]  \textbf{Wolves:} viele Samples bei wenigen anderen Sprechern zugeordnet
\end{itemize}

Dieses schema wurde auf das ergebnis der klassifikation mit dem ibk auf das \textit{mixed test set} angewandt. das ergebnis dieser projektion ist zu sehen in Tabelle:\ref{table:resultsZoo}


\begin{table}[h]
	\centering
    \begin{tabular}{ | l | l | l | l | l |}
    \hline
    Animal & female & male & mixed & Anteil \\ \hline 
    Sheep 	& 32 	& 43	&	75	&	90.36\%	\\ \hline
    Goat	& 2		& 2		& 4 	&	4.82\% \\ \hline
    Lamb	& 1  	& 1		& 1		&	2.41\% \\ \hline
    Wolf	& 0  	& 2 	& 2		&	2.41\% \\ \hline
    \end{tabular}
    \caption{Ergebnisse der Kategorisierung}
   \label{table:resultsZoo}
\end{table}


Das Ergebnis der kategorisierung zeigt dass der großteil unsere Sprecher \textbf{Sheep}'s sind.
Wie zu erwarten wurden unsere falsch authentifizierten Sprecher zu der Kategorie der \textbf{Goat} zugeordnet.




\section{Klassifikation}

in diesem kapitel geht es darum klassifikatoren zu finden die 

bei der Klassifizierung geht es im allg um


Unsere daten sind dabei in die 6 sets undterteilen und beinhalten die feature der recordings.



Zur bestimmung der bessten klassifizieren haben wir die methode des try and eeror verwendet. Das heißt wir haben alle möglichen klassifikatoren in den default setting auf unsere datenbasis angewandt und danach die ergebisse verglichen.

ein gutes ergebnis bestand darin dass möglichst viele samples eines sprechers dem richtigen sprecher zugeordnet wurden. Also der kalssifikator unter verwendung des \textit{mixed test set} eine gute treffergenauigkeit aufwies.\\





dabei hat sich ein klassifikator als besonders gut erwiesen, der ibk. dieser klassifikator erzielte ein ergebnis von 54.94 \% Treffergenauigkeit bei dem 
\textit{mixed test set}.
als vergleich haben wir den klassifikator mit dem zweit besten ergebnis mit aufgeführt, der RandomForest. Die Tabelle:\ref{table:resultsClassifiers} zeigt das Ergebnis beider klassifikatioren mit den dazugehörigen konfiguration des klassifikators. \\


\begin{table}[h]
	\centering
    \begin{tabular}{ | l | l | l |}
    \hline
    Datenset & IBK & RandomForest  \\ \hline 
    female test set & 54.94\%  & 39.14\% \\ \hline
    male test set & 58.50\%  & 41.7679\% \\ \hline
    mixed test set & 53.16\%  & 33.86\% \\ \hline
    \end{tabular}
    \caption{Ergebnisse der Klassifikation}
   \label{table:resultsClassifiers}
\end{table}

Aus den ergebnissen ist zu entnehmen dass es keinen signifikanten unterschied zwischen der den ergebnisse der sets \textit{female teest set} und \textit{male test set} exisitert. der bestehtnde unterschied lässt sich aus der geringen größe des datensets erklären.\\


Ausdem ist zu beobachten dass sich die ergebnisse beioder klassifikatoren verschlechter hat bei erhöhung der anzahl von sprechern.
Dies ist jedoch ein zu erwartendes ergebniss da der klassifikator nun das sample eines sprechers mit 82 anderen samples statt mit 47 bzw. 34 anderen samples verglcihen muss.\\





\subsection{Authentifizierung}

in diesem kaptitel geht es nun um die Aufgabe der \textit{Closed set speaker authentication}. Wobei die akustische Aufnahme eines Sprechers mit der aller anderen möglichen Sprechern verglichen wird und die beste Übereinstimmung als Ergebnis ausgegeben wird.\cite{beigi}
Zu Beachten ist, dass hier im Gegensatz zu der \textit{Open set speaker authentication} es in jedem Fall zu einem Ergebnis kommt.

die aufgabe besteht nun darin das ergebnis der klassifikation der samples zu interpretieren. 
dafür betrachteten für jeden sprecher die verteilung seiner samples.
Hierbei wurde ein sprecher richtig erkannt wenn bei ihm die größte menge samples zugeordnet wurden.
dies bedeutet dass wir auch mit einemgeringen anzahl richtig klassifizierter samples einen sprecher authentifizieren konnten solang die übrigen samples gleichmäßig verteilt waren. 

dieses verfahren wurde bei allen sprechern angewandt und das entstandende ergebniss ist zu sahen in Tabelle:\ref{table:resultsAuth}.



\begin{table}[h]
	\centering
    \begin{tabular}{ | l | l | l | l | l |}
    \hline
    Datenset & Gesamt & Richtig & Falsch & Anteil  	\\ \hline 
    female test set & 35  	& 33	& 2 	& 94.29\%  	\\ \hline
    male test set 	& 48	& 46 	& 2 	& 95.83\% 	\\ \hline
	mixed test set 	& 83  	& 79	& 4		& 95.18\% 	\\ \hline
    \end{tabular}
    \caption{Ergebnisse der Authentifizierung}
   \label{table:resultsAuth}
\end{table}

Das Ergebnis von 95.18\% richtig erkannten sprechern ist gut und entspricht damit dem \textit{state-of-the-art}.\cite{beigi}
Im vergleich dazu wurde im hyke-projekt ein ergebnis von 95\% erreicht, welches mit unseren nahezu identisch ist.\cite{hyke}


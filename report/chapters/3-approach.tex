\section{Unser Ansatz}
\label{approach}

Das Thema unseres Projektes ist die \textbf{Speaker Recognition} , wobei wir anhand von Aufnahmen der Stimme von Personen versuchen diese zu authentifizieren. Genauer definiert ist die Speaker Recognition folgendermaßen:


\textit{Speaker recognition, sometimes referred to as speaker biometrics, includes identification, verification (authentication), classification, and by extension, segmentation, tracking and detection of speakers. It is a generic term used for any procedure which
involves knowledge of the identity of a person based on his/her voice.}
\cite{beigi} \\


In dem Rahmen des Projektes sollten die folgenden Aufgabenstellungen erfolgreich bearbeitet werden:
\begin{itemize}
	\item[\textbullet] Closed set speaker authentication on the Hyke speech database
	\item[\textbullet] Compare the results achieved (in terms of authentication performance) to th results persented in \cite{hyke}
	\item[\textbullet]  A projection of the samples in your data set to the characters of 'Doddingtons Zoo' \cite{zoo}
\end{itemize}
 

Bei den Arbeitsschritte in unserem Projekt haben wir uns an das allgemeine Model für das Authentifizieren von Benutzern anhand von Biometrischen Daten orientiert, welches in \cite{vielhauer} vorgestellt wird. Dieses Model haben wir an unser Thema, der \textit{Speaker Recognition}, und der Aufgabenstellung angepasst:


\begin{itemize}
	\item[\textbullet] Data Acquisition: Verwendung der Hyke Databasis
	\item[\textbullet] Pre-processing: Einteilung der Datenbasis in geschlechtsspezifischen Sets
	\item[\textbullet] Feature Extraction: Merkmals-Extraktion der Audio-Daten mittels AAFE. 
	\item[\textbullet] Post-processing: Aufarbeitung der Feature-Matrizen
	\item[\textbullet] Comparsion and Classification: Klassifizierung der Samples und Authentifizierung der Sprecher
\end{itemize}


Zur Bearbeitung des Projektes wurden uns zwei Programme bereitgestellt:
\begin{itemize}
	\item[\textbullet] \textbf{AAFE} (AMSL Audio Feature Extractor), ist ein Tool f"ur die Extraktion von Merkmalen in Audio-Dateien und entstammt dem AMSL Audio Steganalysis Toolset (AAST).\cite{kraetzer} Anwendung fand das AAFE-Tool in dem Kapitel:\ref{aafe}.
	\item[\textbullet] \textbf{WEKA}, ist eine Sammlung von Algorithmen des Maschinellen Lernens f"ur Aufgaben im Bereich des Data-Mining.\cite{weka} Anwendung fand dieses Tool in den Kapieln:\ref{post} und \ref{classification}.
\end{itemize}


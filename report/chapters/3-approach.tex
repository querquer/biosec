\section{Unser Ansatz}
\label{approach}

es geht um speaker recognition, wobei


\textit{Speaker recognition, sometimes referred to as speaker biometrics, includes identification, verification (authentication), classification, and by extension, segmentation,
tracking and detection of speakers. It is a generic term used for any procedure which
involves knowledge of the identity of a person based on his/her voice.}
\cite{beigi} \\


In diesem rahmen sollten die folgenden aufgabenstellungen erfolgreich abgschlossen werden:
\begin{itemize}
	\item[\textbullet] Closed set speaker authentication on the Hyke speech database
	\item[\textbullet] Compare the results achieved (in terms of authentication performance) to th results persented in \cite{hyke}
	\item[\textbullet]  A projection of the samples in your data set to the characters of 'Doddingtons Zoo' 
\end{itemize}
 

Bei den Arbeitsschritte in unserem Projekt haben wir uns an das allgemeine Model für das Authentifizieren von Benutzern anhand von Biometrischen Daten orientiert, welches in \cite{vioelhauer} vorgestellt wird. Dieses Model haben wir an unser Thema, der \textit{Speaker Recognition}, und der Aufgabenstellung angepasst.

\begin{itemize}
	\item[\textbullet] Data Acquisition: Hyke Database \ref{database}
	\item[\textbullet] Pre-processing: kein klassisches pre-pro..., einteilung in Sets \ref{pre}
	\item[\textbullet] Feature Extraction: Merkmalsentnahme der Audioaufnahmen mittels AAFE. \ref{aafe}
	\item[\textbullet] Post-processing: eingefügt \ref{post}
	\item[\textbullet] Comparsion and Classification: \ref{classification}
\end{itemize}


die vorgestellten Ablauf eines Authentifizieungsprozesses 

Bei unseren arbeitsschritten haben wir uns dabei an
an der von ... vorgestellen pro chain gehalten

General model for biometric user authentication
\cite{vielhauer}


dieser prozesskette haben wir noch den schritt des post-processing hinzugefügt, 
weil das entfernen von noise aus feature mit weka angeboten hat.


für diese aufgabe wurden uns die folgenden Programme bereitgestellt:

\begin{itemize}
	\item[\textbullet] \textbf{AAFE} (AMSL Audio Feature Extractor), ist ein Tool f"ur die Extraktion von Merkmalen in Audio-Dateien und entstammt dem AMSL Audio Steganalysis Toolset (AAST).\cite{kraetzer} Anwendung fand das AAFE-Tool in dem Kapitel:\ref{aafe}.
	\item[\textbullet] \textbf{WEKA}, ist eine Sammlung von Algorithmen des Maschinellen Lernens f"ur Aufgaben im Bereich des Data-Mining.\cite{weka} Anwendung fand dieses Tool in den Kapieln:\ref{post} und \ref{classification}.
\end{itemize}


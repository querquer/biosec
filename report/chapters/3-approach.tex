\section{Our Approach}
\label{approach}


aufgabenstellung \ref{task}
\begin{itemize}
	\item[\textbullet] Closed set speaker authentication on the Hyke speech database
	\item[\textbullet]  Compare the results achieved (in terms of authentication performance) to th results persented in \cite{hyke}
	\item[\textbullet]  A projection of the samples in your data set to the characters of 'Doddingtons Zoo' 
\end{itemize}

def. speaker recognition \cite{beigi}
Speaker recognition, sometimes referred to as speaker biometrics, includes identification, verification (authentication), classification, and by extension, segmentation,
tracking and detection of speakers. It is a generic term used for any procedure which
involves knowledge of the identity of a person based on his/her voice.

Bei unseren arbeitsschritten haben wir uns dabei an
an der von ... vorgestellen pro chain gehalten
\begin{itemize}
	\item kommt woher? \cite{dittmann}
	\item genaue erklärung in den einzelnen kapiteln
\end{itemize}


für diese aufgabe wurden uns die folgenden Programme bereitgestellt:

\begin{itemize}
	\item[\textbullet]  AAFE, "u ist ein ... und wurde für die Feature Extraction verwendet\ref{aafe}
	\item[\textbullet]  WEKA, ist ein ... und wurde für die ref{post} und ref{classifikation} verwendet
\end{itemize}


\section{Nachtverarbeitung}
\label{post}
Die extrahierten Daten wurden mit Hilfe von Weka aufbereitet. Dadurch sollten bessere Ergebnisse bei der Klassifikation erzeugt werden. Dazu wurde die Features \textit{lbs flipping ratio} (in allen Instanzen 922337203685477.6000) und \textit{lbs flipping rate} (in allen Instanzen 0) entfernt. Weil sie in allen Fällen gleich sind lassen sich an ihnen keine Unterschiede in den Aufnahmen feststellen.

In den Aufnahmen gibt es Bereiche die keine Stimme enthalten. Diese konzentrieren sich auf Anfang und Ende der Datei. Es gibt auch Pausen zwischen den gesprochenen Ziffern. Die \grqq stillen\grqq  Bereiche enthalten keine Information über die Stimme und somit den Sprecher. Dadurch wird die spätere Klassifikation erschwert. Um die \grqq Stille\grqq  herauszufiltern wurden alle Samples mit einer geringen Amplitude gelöscht. Dazu wurde das Feature \textit{rms amplitude} genutzt und alle Samples mit einem Wert unter 10 gefiltert. Es wurde der \textit{RemoveWithValues} Filter von Weka mit den Parametern -S 10.0 -C 5 -L first-last verwendet. Dadurch wurden von 50.424 Samples 32.026 entfernt. Das heißt es wurden rund 64 Prozent der Datenbasis entfernt.

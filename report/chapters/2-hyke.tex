\section{Hyke-System}
\label{hyke}

Da sich unter projekt von dem des Hyke ableiten wollen wir zuerst einmal im Folgenden das Hyke Projekt vorstellen.

Das Hyke-System hat seinen Ursprung in der Region Rajasthan in Nordwest Indien. In dieser ländlichen Region sind ausschließlich kleine Schulen anzufinden die meist aus 1 bis 3 Klassen bestehen. Dennoch unterstehen sie der behördlichen Bildungseinrichtung der Region die sicherstellen möchte, dass in den Schulen ein regelmäßiger Unterricht stattfindet. \\

Das bisherige System basierte auf eine visuelle Überprüfung der Anwesenheit. Dazu waren die Lehrer angehalten zwei mal täglich ein Bild von sich aufzunehmen und an die zentrale Einrichtung zu schicken. Dort wurde dieses Bild von Mitarbeitern verwendet um eine manuelle Authentifizierung vorzunehmen.
Der Nachteil dieses Systems bestand in den hohen Kosten die es erzeugte, zum einen durch das Vorhandensein einer Kamera an jedem Standort und zum anderen durch die manuelle Überprüfung der Aufnahmen.\\

Aus diesem Grund entschloss man sich für die Entwicklung von Hyke. Ein System welches die Lehrer anhand ihrer Stimme authentifiziert. Dies hat den Vorteil, dass kaum Anschaffungen der Aufnahmegeräte von Nöten waren, da bereits 75\% der Schulen über ein Telefon verfügten. Au\ss{}erdem ist nun eine automatisierte Authentifizierung der Lehrer möglich.Lehrer möglich sein.\\



\documentclass{acmtog} 

\usepackage{german} 
\usepackage[colorlinks=true, urlcolor=blue, linkcolor=blue, citecolor=black]{hyperref}


\acmVolume{1}
\acmNumber{1}
\acmYear{2015}
\acmMonth{1}
\acmArticleNum{}
\acmdoi{}

\begin{document}

\title{Biometrics and Security\\ Speaker Recognition} 

\author{Jonas Marquardt {\upshape und} Maik Riestock
\affil{Otto von Guericke University Magdeburg - Advanced Multimedia and Secruity Lab (AMSL)}}


\keywords{Speaker Recognition, Audio Feature Extraction, Doddingtons Zoo}


\maketitle

\begin{bottomstuff}
This report was created in the context of the course “Biometrics and Security [BIOSEC]” winter term 2014/15.
This course was held by: Prof. Dr.-Ing. Jana Dittmann and Prof. Dr.-Ing. Claus Vielhauer; Research group Multimedia and Security, Otto-von-Guericke-University of Magdeburg, Germany.
The course was supported by: Dr.-Ing. Christian Krätzer, M.Sc. Kun Qian
\end{bottomstuff}


\begin{abstract}
speaker reco\\
orientiert am hyke(database)\\
\textit{Closed set speaker authentication}\\
projektion auf 'Doddingtons Zoo'\\

\end{abstract}

\section{Motivation}

Um Personen eindeutig zu identifizieren gibt es verschiedene Verfahren. Eines davon ist die Erkennung der Stimme. Als Aufnahmegerät ist ein handelsübliches Mikrophone ausreichend. In dieser Übung ging es darum, herauszufinden wie eine Stimmenerkennung umgesetzt wird und welche Eigenheiten dieses Verfahren mit sich bringt.
  		% Jonas
\section{Hyke Projekt}


\begin{itemize}
	\item allgemeine informationen
	\item verbidung zu unserem
	\item unterschiede zu unserem
\end{itemize}


				% Maik
\section{Our Approach}
\label{approach}


aufgabenstellung \ref{task}
\begin{itemize}
	\item[\textbullet] Closed set speaker authentication on the Hyke speech database
	\item[\textbullet]  Compare the results achieved (in terms of authentication performance) to th results persented in \cite{hyke}
	\item[\textbullet]  A projection of the samples in your data set to the characters of 'Doddingtons Zoo' 
\end{itemize}

def. speaker recognition \cite{beigi}
Speaker recognition, sometimes referred to as speaker biometrics, includes identification, verification (authentication), classification, and by extension, segmentation,
tracking and detection of speakers. It is a generic term used for any procedure which
involves knowledge of the identity of a person based on his/her voice.

Bei unseren arbeitsschritten haben wir uns dabei an
an der von ... vorgestellen pro chain gehalten
\begin{itemize}
	\item kommt woher? \cite{dittmann}
	\item genaue erklärung in den einzelnen kapiteln
\end{itemize}


für diese aufgabe wurden uns die folgenden Programme bereitgestellt:

\begin{itemize}
	\item[\textbullet]  AAFE, "u ist ein ... und wurde für die Feature Extraction verwendet\ref{aafe}
	\item[\textbullet]  WEKA, ist ein ... und wurde für die ref{post} und ref{classifikation} verwendet
\end{itemize}

			% Maik
\section{Datenbasis}
\label{database}
Die Datenbasis wurde dem Hyke-Projekt entnommen. Sie kann unter folgender URL heruntergeladen werden:
%TODO: URL einfügen
Sie umfasst Aufnahmen von 83 verschiedenen Sprechern, davon 48 männlich und 35 weiblich. Von jeder Person gibt es fünf Aufnahmen in denen Abfolgen verschiedener Ziffern gesprochen werden. Die Sprache dabei ist Englisch. Die Länge der Aufnahmen liegt zwischen 5 und 35 Sekunden. Es gibt auch Aufnahmen, die keine Stimme enthalten. %TODO ändern
Die Stimmen wurden über das Telefon aufgenommen und bieten daher eine geringere Bandbreite als die menschliche Stimme hat. Bei den Sprechern handelt es sich um Inder mit verschiedenen Hintergründen. Die Aufnahmen enthalten teilweise Hintergrundgeräusche, vom leisem Rauschen bis zu Gesprächen und Musik. 
			% Jonas
\section{Pre-Processing}
\label{pre}

In diesem Kapitel geht es um die Daten unserer Datenbank auf die folgenden schritte vorzubereiten.

Da wir in unseren Ergebnissen am ende einen m"oglichen Unterschied zwischen den Ergebnissen der Authentifizierung beider Geschlechtern beobachten zu können, wurden die Datenbank in sechs Sets unterteilt.
Hierf"ur wurde das Set mit Sprechern von beiden Geschlechtern, \textit{mixed set}, aufgeteilt in zwei Sets mit ausschließlich Stimmen von weiblichen Sprechern, \textit{female set}, und mit ausschließlich männlichen Sprechern, \textit{male set}. 


Zus"atzlich ben"otigen die Klassifikatoren zwei verschiedene Sets von Daten.
mit dem einen Set wird das Model trainiert, hier \textit{train set}, und mit dem andren Set evaluiert, hier \textit{test set}. \\



Die Datenbasis aufgeteilt in folgende Sets:
\begin{itemize}
    \item[\textbullet] mixed train set
    \item[\textbullet] mixed test set
    \item[\textbullet] female train set
    \item[\textbullet] female test set
    \item[\textbullet] male train set
    \item[\textbullet] male test set
\end{itemize}
				% Maik
\section{Feature Extraction}

\begin{itemize}
	\item Benutzung von AAFE
    \item Woher kommt er?
    \item Was macht er?
    \item Wie?
\end{itemize}



		% Jonas
\section{Post-Processing}
Die Features wurden mit Hilfe von Weka aufbereitet. Dadurch sollten bessere Ergebnisse bei der Klassifikation erzeugt werden. Dazu wurde die Features "lbs_flipping_ratio" (in allen Instanzen 922337203685477.6000) und "lbs_flipping_rate" (in allen Instanzen 0) entfernt. Weil sie in allen Fällen gleich sind lassen sich an ihnen keine Unterschiede in den Aufnahmen feststellen.

In den Aufnahmen gibt es Bereiche die keine Stimme enthalten. Diese konzentrieren sich auf Anfang und Ende der Datei. Es gibt auch Pausen zwischen den gesprochenen Ziffern. Die "stillen" Bereiche enthalten keine Information über die Stimme und somit den Sprecher. Dadurch wird die spätere Klassifikation erschwert. Um die "Stille" herauszufiltern wurden alle Samples mit einer geringen Amplitude gelöscht. Dazu wurde das Feature "rms_amplitude" genutzt und alle Samples mit einem Wert unter 10 gefiltert. Es wurde der "RemoveWithValues" Filter von Weka mit den Parametern -S 10.0 -C 5 -L first-last verwendet.

\begin{itemize}
    \item mit welchem ergebnis?
\end{itemize}

				% Jonas
\section{Klassifikation}
\label{classification}

in diesem kapitel geht es um die klassifikation unserer daten.

Klassifizierung ist def durch


Für diese Aufgabe haben wir das tool WEKA verwendet, welches in Kapitel:\ref{approach} vorgestellt wurde.

Usnere ausgangsdaten für die klassifikatoren sind die aufbearbeiteten features, welche in Kapitel:\ref{post} vorgestellt wurden, die wir in den folgenden sets unterscheiden:
\begin{itemize}
    \item[\textbullet] mixed train set
    \item[\textbullet] mixed test set
    \item[\textbullet] female train set
    \item[\textbullet] female test set
    \item[\textbullet] male train set
    \item[\textbullet] male test set    
\end{itemize}



Zur bestimmung der bessten klassifizieren haben wir die methode \textit{try and error} verwendet. Das heißt wir haben alle anwendbaren klassifikatoren in der Standarteinstellung auf unsere datenbasis angewandt und danach die ergebisse verglichen.

ein gutes ergebnis bestand darin dass möglichst viele samples eines sprechers dem richtigen sprecher zugeordnet wurden. Also der kalssifikator unter verwendung des \textit{female/male/mixed test set} eine gute treffergenauigkeit aufwies.\\





dabei hat sich ein klassifikator als besonders gut erwiesen, der ibk. dieser klassifikator erzielte ein ergebnis von 54.94 \% Treffergenauigkeit bei dem 
\textit{mixed test set}.
als vergleich haben wir den klassifikator mit dem zweit besten ergebnis mit aufgeführt, der RandomForest. Die Tabelle:\ref{table:resultsClassifiers} zeigt das Ergebnis beider klassifikatioren mit den dazugehörigen konfiguration des klassifikators. \\


\begin{table}[h]
	\centering
    \begin{tabular}{ | l | l | l |}
    \hline
    Datenset 		& IBK & RandomForest  \\ \hline 
    female test set & 54.94\%  & 39.14\% \\ \hline
    male test set 	& 58.50\%  & 41.7679\% \\ \hline
    mixed test set 	& 53.16\%  & 33.86\% \\ \hline
    Konfiguration	& -K 1 -W 0 -A & -I 10 -K 0 -S 1 \\ \hline
    \end{tabular}
    \caption{Ergebnisse der Klassifikation}
   \label{table:resultsClassifiers}
\end{table}

Aus den ergebnissen ist zu entnehmen dass es keinen signifikanten unterschied zwischen der den ergebnisse der sets \textit{female teest set} und \textit{male test set} exisitert. der bestehtnde unterschied lässt sich aus der geringen größe des datensets erklären.\\


Außerdem ist zu beobachten dass sich die Ergebnisse beioder klassifikatoren verschlechter hat bei erhöhung der anzahl von sprechern.
Dies ist jedoch ein zu erwartendes ergebniss da der klassifikator nun das sample eines sprechers mit 82 anderen samples statt mit 47 bzw. 34 anderen samples verglcihen muss.\\





\subsection{Authentifizierung}
\label{auth}

in diesem kaptitel geht es nun um die Aufgabe der \textit{Closed set speaker authentication}. Wobei die akustische Aufnahme eines Sprechers mit der aller anderen möglichen Sprechern verglichen wird und die beste Übereinstimmung als Ergebnis ausgegeben wird.\cite{beigi}
Zu Beachten ist, dass hier im Gegensatz zu der \textit{Open set speaker authentication} es in jedem Fall zu einem Ergebnis kommt.

die aufgabe besteht nun darin das ergebnis der klassifikation der samples zu interpretieren. 
dafür betrachteten für jeden sprecher die verteilung seiner samples.
Hierbei wurde ein sprecher richtig erkannt wenn bei ihm die größte menge samples zugeordnet wurden.
dies bedeutet dass wir auch mit einemgeringen anzahl richtig klassifizierter samples einen sprecher authentifizieren konnten solang die übrigen samples gleichmäßig verteilt waren. 

dieses verfahren wurde bei allen sprechern angewandt und das entstandende ergebniss ist zu sahen in Tabelle:\ref{table:resultsAuth}.



\begin{table}[h]
	\centering
    \begin{tabular}{ | l | l | l | l | l |}
    \hline
    Datenset & Gesamt & Richtig & Falsch & Anteil  	\\ \hline 
    female test set & 35  	& 33	& 2 	& 94.29\%  	\\ \hline
    male test set 	& 48	& 46 	& 2 	& 95.83\% 	\\ \hline
	mixed test set 	& 83  	& 79	& 4		& 95.18\% 	\\ \hline
    \end{tabular}
    \caption{Ergebnisse der Authentifizierung}
   \label{table:resultsAuth}
\end{table}

Das Ergebnis von 95.18\% richtig erkannten sprechern ist gut und entspricht damit dem \textit{state-of-the-art}.\cite{beigi}
Im vergleich dazu wurde im hyke-projekt ein ergebnis von 95\% erreicht, welches mit unseren nahezu identisch ist.\cite{hyke}

	% Maik
\section{Doddingtons Zoo}


Doddingtons Zoo
geht es darum dass sprecher ein unterschiedliches verhalten bezüglich den erfolg ihrer authentifikation aufzeigen. \cite{zoo}
dabei lassen sich sprecher in vier kategoriern unterscheiden die jeweils von einem tier representiert werden. \\

Eine Beschreibung dieser Kategorieren:
\begin{itemize}
	\item \textbf{Sheeps:} außerordentlich \textit{leicht} von dem System erkannt, die Mehrheit der Sprecher gehört dieser Kategorie an
	\item \textbf{Goats:} außerordentlich \textit{schwer} von dem System erkannt
	\item \textbf{Lambs:} außerordentlich \textit{verwundbar} gegenüber Nachahmung
	\item \textbf{Wolves:} außerordentlich \textit{erfolgreich} bei der Nachahmung anderen Sprecher
\end{itemize}
\cite{dittmann}


nun war es die aufgabe war nun eine projektion dieser kategorien auf unsere klassifizierungs ergebnisse. als entscheidungsgrundlage waren nun nicht nur  die erfolgreich klassifizierten samples intressant sondern auch die verteilung der samples die nicht dem richigen sprecher zugeordnet wurden.

Eine Beschreibung dieser Kategorieren:
\begin{itemize}
	\item \textbf{Sheeps:} viele richtig klassifizierte Samples, 
	\item \textbf{Goats:} wenig richtig klassifizierte Samples
	\item \textbf{Lambs:} viele Sample von anderen Sprechern wurden diesem Sprecher zugeordnet
	\item \textbf{Wolves:} außerordentlich \textit{erfolgreich} bei der Nachahmung anderen Sprecher
\end{itemize}



\begin{table}[h]
	\centering
    \begin{tabular}{ | l | l | l | l | l |}
    \hline
    Animal & female & male & mixed & Anteil \\ \hline 
    Sheep 	& 32 	& 43	&	75	&	90.36\%	\\ \hline
    Goat	& 2		& 2		& 4 	&	4.82\% \\ \hline
    Lamb	& 1  	& 1		& 1		&	2.41\% \\ \hline
    Wolf	& 0  	& 2 	& 2		&	2.41\% \\ \hline
    \end{tabular}
    \caption{Ergebnisse der Kategorisierung}
   \label{table:resultsZoo}
\end{table}


interpretation


\begin{itemize}
	\item Vorstellung des Zoos
	\item Anwendung
	\item Results
\end{itemize}				% Maik
\section{conclusion}
Es ist möglich einen Menschen anhand seiner Stimme zu identifizieren. Dies eröffnet Anwendungsbereiche, die mit anderen biometrischen Verfahren nicht möglich sind. Ein Beispiel ist die Identifizierung einer Person über das Telefon. Wir konnten in unseren Experimenten, mit geringem Aufwand, 96,68 Prozent der Personen eindeutig Identifizieren. Bei den 3,32 Prozent der nicht identifizierten lag eine schlechte Datenbasis vor. Das heißt der Erfolg bei der Identifizierung hängt signifikant von der Datenbank ab.		% Jonas
\section{Future Work}


\begin{itemize}
	\item future stuff
\end{itemize}			% Maik

%--------------------------------------------------------------------------------------
\newpage
% Bibliography
\bibliographystyle{ACM-Reference-Format-Journals}
\bibliography{speakerreco}

\listoftables
\appendix
\section{Task Describtion}
\label{task}

Run your prototype on the collected data and perform a performance evaluation with your prototype. The evaluation must include:

The evaluation must include:
\begin{itemize}
	\item Closed set speaker authentication on the Hyke speech database
	\item Compare the results achieved (in terms of authentication performance) to th results persented in 
	\item A projection of the samples in your data set to the characters of 'Doddingtons Zoo' 
\end{itemize}




\end{document}


\section{The framework AnnoWaNo}

AnnoWaNO is a framework for annotation watermarking and were be developed by the research group multimedia and security by the university of magdeburg. Annotation watermarking denotes a specific sort of watermarking where information in the whole image or a part of the image are embedded in the image itself. The framework include three algorithm for the embedding process, the WetPaperCode, the Block-Luminance and the Dual Domain DFT. For the evaluation, we only consider the DDD since this algorithm, since this is the actual new development and the other two algorithms are used only as reference. The embedded information consist of three parts, semantic, shape and the relationship to other annotations. The algorithm itself operates in the frequency domain using a DFT on blocks. \cite{Schott2009} The semantic , most often a text string, is embedded in the phase part, whereas the relationship, described as a vector, is embedded in the magnitude part.



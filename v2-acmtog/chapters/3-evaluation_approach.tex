\section{Evaluation Approach}


\begin{itemize}
	\item what is evaluation?
	\item what does evaluation mean in this context?
	\item what is our goal?
	\item how do we want to reach this goal?
\end{itemize}



The following subsection contain a detailed description of the expected errors, the framework parameters and the used attacking scenarios.
 
\subsection{Expected errors}

\textit{missing annotation:}
This error indicates whether an annotation has been successfully retrieved out of an image.
This is the most important aspect, because an unsuccessful retrieving means that the information contained in the image is lost. 
This error will be evaluated by the results of the frameworks, which stands for disposal after the process of retrieving and includes the number of successful retrieved annotation.
\textbf{review needed: Satzanfang}

\textit{area missmatch:}
Area refers to the region in which an annotation is embedded in the image. An error of the area evolves in the process of retrieving and describes the difference of the area during annotation and after the process of retrieving. This error is evaluated by a visual comparison between the annotated and retrieved image.
\textbf{review needed}

\subsection{Framework parameters}

The described end here parameter have strong influence on the Dual Domain DFT and thus to the embedding process and are adjustable via framework. The two most important parameters are embedding strength and block size, and are described in more detail below.
\textbf{review needed}

\textit{embedding strength:}
The embedding strength is a parameter which is used to embedded the information of the relationship of annotations and is regarded as the minimum value a magnitude for a frequency used for embedding must have. The higher the magnitude the more robust against changes the frequency is.\cite{Vielhauer2009}
\textbf{review needed}
   
\textit{block size:}
This parameter describes the size of the blocks used for the Dual Domain DFT for the embedding process. as the block size directly influences the count of blocks, a higher bock size lowers block count, but increase the capacity.\cite{Vielhauer2009}
\textbf{review needed}
 
\subsection{Attack scenarios}

\begin{itemize}
    \item for the report we tried different attack scenarios
    \item two (cropping and noise attacks) were given in the task description
    \item for the remaining attacks we chose:
    \begin{itemize}
        \item scaling/resizing (first step: smaller, second step: original size)
        \item filetype conversions (gif, png, tif, jpg)
        \item converting to grayscale
        \item adding a filter (vintage), downloaded script [TODO]
    \end{itemize}
\end{itemize}

All attacks (except for cropping) were done with the tool ImageMagick [TODO] using the following commands:

Adding impulse noise to the image:
\begin{verbatim}
convert input.bmp +noise Impulse output.bmp 
\end{verbatim}

Resizing the image was done in two steps. In the first step, the image was resized to be slightly smaller than the 
original image and in the second step that smaller image was resized back to its original dimensions.
For the coin image the command looks like this:
\begin{verbatim}
convert coins.bmp -resize 4126x2321 coins_resized1.bmp
convert coins_resized1.bmp -resize 4128x2322 coins_resized2.bmp
\end{verbatim}

Filetype conversion
\begin{verbatim}
convert input.bmp output.gif
\end{verbatim}

Grayscale conversion
\begin{verbatim}
convert input.bmp -colorspace Gray output.bmp
\end{verbatim}

vintage filter
\begin{verbatim}
vintage2 input.bmp output.bmp
\end{verbatim}
